%%%%%%%%%%%%%%%%%%%%%%%%%%%%%%%%%%%%%%%%
% Classe do documento
%%%%%%%%%%%%%%%%%%%%%%%%%%%%%%%%%%%%%%%%

% Opções:
%  - Graduação: bacharelado/engenharia/licenciatura
%  - Pós-graduação: mestrado/doutorado, ppca/ppginf

\documentclass[engenharia]{UnB-CIC}%
% \documentclass[mestrado,ppca]{UnB-CIC}%

\usepackage{pdfpages}% incluir PDFs, usado no apêndice
%\usepackage{indentfirst} % Faz com que todos os parágrafos (inclusive o primeiro) recebam recuo



%%%%%%%%%%%%%%%%%%%%%%%%%%%%%%%%%%%%%%%%
% Informações do Trabalho
%%%%%%%%%%%%%%%%%%%%%%%%%%%%%%%%%%%%%%%%
\orientador{\prof \dr Alexandre Zaghetto}{CIC/UnB}%
%\coorientador{\prof \dr José Ralha}{CIC/UnB}
\coordenador{\prof \dr Ricardo Zelenovsky}{ENE/FT/UnB}%
\diamesano{24}{dezembro}{2015}%

\membrobanca{\prof \dr Banca 1}{Universidade de Brasília}%
\membrobanca{\prof \dr Banca 2}{Universidade de Brasília}%

\autor{Mateus Mendelson E. da}{Silva}%

\titulo{Técnica anti-spoofing para leitores biométricos 3D de impressões digitais sem toque}%

\palavraschave{biometria, impressão, digital, anti, spoofing, LBP, trabalho de conclusão de curso, LISA}%
\keywords{biometrics, fingerprint, anti, spoofing, LBP, thesis, LISA}%

\CDU{004}%

\newcommand{\unbcic}{\texttt{UnB-CIC}}%

%%%%%%%%%%%%%%%%%%%%%%%%%%%%%%%%%%%%%%%%
% Texto
%%%%%%%%%%%%%%%%%%%%%%%%%%%%%%%%%%%%%%%%
\begin{document}%
	\nocite{*} % Permite que todas as referências bibliográficas sejam exibidas mesmo que não sejam referenciadas no texto

	\capitulo{1_Introducao}{Introdução}%
    \capitulo{2_Spoofing}{Spoofing}%
    \capitulo{3_Imagens}{Técnicas e Manipulação de Imagens}%
    \capitulo{1_Introducao_original}{Introdução}%
    \capitulo{2_UnB-CIC}{A Classe \unbcic}%
    \capitulo{3_TCC}{Trabalho de Conclusão de Curso}%
    \capitulo{4_Apresentacao}{Apresentações}%

    \apendice{Apendice_Fichamento}{Fichamento de Artigo Científico}%
    \anexo{Anexo1}{Documentação Original \unbcic\ (parcial)}%
\end{document}%