Este capítulo visa apresentar os conceitos e conhecimentos iniciais necessários ao entendimento dos jargões, classificações e técnicas comumente utilizados no contexto de biometria. Caso seja de seu interesse, o leitor é encorajado a buscar informações complementares nos ítens bibliográficos que o auxiliem no entendimento e/ou aprofundamento do conteúdo aqui apresentado.

%%%%%%%%%%%%%%%%%%%%%%%%%%%%%%%%%%%%%%%%%%%%%%%%%%%%%%%%%%%%%%%%%%%%%%%%%%%%%%%%
%%%%%%%%%%%%%%%%%%%%%%%%%%%%%%%%%%%%%%%%%%%%%%%%%%%%%%%%%%%%%%%%%%%%%%%%%%%%%%%%
%%%%%%%%%%%%%%%%%%%%%%%%%%%%%%%%%%%%%%%%%%%%%%%%%%%%%%%%%%%%%%%%%%%%%%%%%%%%%%%%
\section{Biometria}%
A capacidade de autenticar indivíduos de forma eficiente e precisa tem se tornado um requisito cada vez mais imprescindível aos sistemas computacionais. Seu uso varia desde aplicações comuns ao nosso dia-a-dia (como o desbloquear da tela de um celular por meio de impressão digital) até aplicações que exigem maiores níveis de segurança (tais como controle de acesso a determinados espaços físicos, informações sigilosas e objetos de valor). Sendo assim, é necessário escolher características intrínsecas à cada indivíduo e elaborar técnicas que sejam capazes de extraí-las.

O processo de autenticação pode ser feito por meio de três credenciais, sendo elas: a posse de um objeto específico (como um cartão RFID usado para acessar uma sala); o conhecimento de certa informação (como a senha de uma conta bancária); ou a presença de certa característica (como uma impressão digital) \cite{vital_signs}. Este último é o foco do trabalho aqui apresentado e deve ser tomado como o objeto de estudo da biometria.

Define-se biometria como sendo a identificação automatizada de um indivíduo a partir de suas características comportamentais e fisiológicas que sejam únicas e de imitação não trivial \cite{guidelines_practices, clarke}.

Características comportamentais são aquelas relacionadas ao modo de agir de uma pessoa. Já as características fisiológicas são aquelas relacionadas à estrutura física de uma pessoa.

A fim de ilustrar as definições acima, os seguintes traços biométricos são assim classificados \cite{biometric_systems, introducao_biometria}:

\begin{enumerate}
	\item Comportamentais
	\begin{itemize}
		\item Assinatura
		\item Marcas de pressão ao escrever
		\item Voz
		\item Modo de digitar
		\item Modo de andar
	\end{itemize}

	\item Fisiológicas
	\begin{itemize}
		\item Impressões digitais
		\item Mãos
		\item Face
		\item Íris
		\item Retina
		\item Formato dos dedos
		\item DNA
	\end{itemize}
\end{enumerate}

É fácil notar que, embora cada uma dessas características seja intrínseca ao indivíduo, um sistema baseado em características comportamentais tende a ser mais vulnerável do que outro que se baseie em características fisiológicas. Traços comportamentais são facilmente observáveis e, muitas vezes, podem ser copiados por outros seres humanos, pois se tratam de sequências de ações executadas de acordo com o padrão observado. Como exemplo, pode-se citar os corriqueiros e bem conhecidos casos de falsificação de assinaturas. Traços fisiológicos, por sua vez, são características físicas, não aprendidas. Para que se possa burlar tal sistema, é necessário o uso de materiais e técnicas específicos para a confecção das cópias falsas.

%%%%%%%%%%%%%%%%%%%%%%%%%%%%%%%%%%%%%%%%%%%%%%%%%%%%%%%%%%%%%%%%%%%%%%%%%%%%%%%%
%%%%%%%%%%%%%%%%%%%%%%%%%%%%%%%%%%%%%%%%%%%%%%%%%%%%%%%%%%%%%%%%%%%%%%%%%%%%%%%%
\subsection{Características de traços biométricos} \label{caracteristicas}
Diante de tantas características presentes nos seres humanos, é necessário definir os pré-requisitos para que uma simples característica possa vir a ser utilizada como traço biométrico.

Traços biométricos devem possuir os seguintes requisitos \cite{an_introduction_biometric_recognition}:

\begin{enumerate}
	\item \textbf{Universalidade}: todos os usuários do sistema devem possuir tal característica;
	\item \textbf{Unicidade}: a característica deve ser única em cada indivíduo, ou seja, quaisquer dois indivíduos não devem ser capazes de apresentar a mesma formação para a característica em questão. Por exemplo, impressões digitais são uma característica que possui formações distintas em cada pessoa.
	\item \textbf{Permanência}: a característica não deve ser variável, fornecendo sempre resultados suficientemente iguais em medições feitas em quaisquer instantes no tempo.
	\item \textbf{Viabilidade de coleta}: é necessário que seja possível a obtenção de medições da característica. 
\end{enumerate}

Sendo atendidos os requisitos acima, ainda é necessário avaliar se, em termos práticos, a coleta de suas medições é viável.

Primeiramente, é preciso avaliar a aceitabilidade por parte dos usuários. Devem ser considerados o modo com o qual as medições serão feitas (conforto), questões de privacidade (desejo do usuário de fornecer ou não tais medições), ética, etc.

O quão fácil (ou difícil) seria para que um invasor conseguisse imitar o traço biométrico e burlar o sistema também é um ponto que requer atenção, pois de nada serve um traço único que pode ser facilmente copiado.

Por se tratar de uma abordagem computacional, é de grande importância avaliar a performance do sistema. O tempo tomado durante as medições e no processo de busca e autenticação são fatores de alto impacto, bem como a precisão e os recursos utilizados.

Dentre as características aceitas como traços fisiológicos, impressões digitais, face e íris são as mais utilizadas e aceitas nos sistemas biométricos atuais \cite{guidelines_practices}, pois estas alcançam níveis aceitáveis nos parâmetros aqui apresentados.

%%%%%%%%%%%%%%%%%%%%%%%%%%%%%%%%%%%%%%%%%%%%%%%%%%%%%%%%%%%%%%%%%%%%%%%%%%%%%%%%
%%%%%%%%%%%%%%%%%%%%%%%%%%%%%%%%%%%%%%%%%%%%%%%%%%%%%%%%%%%%%%%%%%%%%%%%%%%%%%%%
%%%%%%%%%%%%%%%%%%%%%%%%%%%%%%%%%%%%%%%%%%%%%%%%%%%%%%%%%%%%%%%%%%%%%%%%%%%%%%%%
\section{Impressão Digital}
Impressão digital é o nome dado à marca formada pelo conjunto de dobras e vales presentes na ponta de cada dedo (falange). Essa marca é única para cada indivíduo e é imutável, ou seja, mesmo com o passar do tempo esse traço permanece o mesmo. Utilizando os conceitos na Seção \ref{caracteristicas}, vê-se que se trata de uma característica que possui unicidade, permanência, é universal e de fácil coleta, e, por isso, a impressão digital é um dos traços biométricos mais aceitos e utilizados no mundo. Ela é uma característica tão única que nem mesmo irmãos gêmeos possuem digitais iguas. Tais marcas podem ser vistas a olho nu se observadas com atenção, como mostrado na \refFig{fingerprint}.%

Há relatos na literatura de que, com o passar dos anos, algumas digitais podem acabar sendo confundidas como sendo falsas por alguns leitores biométricos. Isso não significa que as impressões digitais tenham se alterado, mas que sua qualidade sofre com o passar o tempo. Tal problema, entretanto, já possui algumas soluções, como pode ser visto em  \cite{modular_architecture}.

\figura{fingerprint}{Exemplo de impressão digital}{fingerprint}{width=.45\textwidth}%

%%%%%%%%%%%%%%%%%%%%%%%%%%%%%%%%%%%%%%%%%%%%%%%%%%%%%%%%%%%%%%%%%%%%%%%%%%%%%%%%
%%%%%%%%%%%%%%%%%%%%%%%%%%%%%%%%%%%%%%%%%%%%%%%%%%%%%%%%%%%%%%%%%%%%%%%%%%%%%%%%
%%%%%%%%%%%%%%%%%%%%%%%%%%%%%%%%%%%%%%%%%%%%%%%%%%%%%%%%%%%%%%%%%%%%%%%%%%%%%%%%
\section{Sistemas Biométricos}
Sistemas biométricos são os responsáveis por coletar as medições dos traços biométricos utilizados como credencial de acesso e decidir se o usuário possui as devidas permissões. Muitas aplicações fazem uso de mais de um traço biométrico ao mesmo tempo. Idealmente, a aquisição das medições dos traços deve ser realizada em apenas uma única interação com o usuário. Entretanto, por se tratar do objetivo deste trabalho, apenas sensores biométricos para impressões digitais serão abordados.

O processo de autenticação de credenciais é dividido em etapas, sendo elas \cite{2D_3D_survey, introducao_biometria}: aquisição de credenciais, pré-processamento das credenciais fornecidas, extração de pontos de interesse (\textit{feature extraction}), comparação com as credenciais existentes no banco de dados (\textit{matching}) e tomada de decisão.

O escopo de cada etapa e a ordem nas quais são executadas se definem da seguinte maneira:

\begin{enumerate}
	\item \textbf{Aquisição de credenciais}: esta etapa é iniciada no engajamento\footnote{Defini-se \emph{engajamento} como sendo a ação na qual o usuário interage com o sistema biométrico com o intuito de fornecer credenciais (sejam elas verdadeiras ou falsas).} e seu objetivo é a obtenção das medições biométricas. Este processo deve ser capaz de obter amostras com qualidade suficiente e, ainda, manter sua viabilidade, ou seja, o conforto do usuário deve ser levado em consideração. Há três principais métodos de aquisição, que serão tratados nas Seções \ref{swipe}, \ref{touch} e \ref{touchless}.
	\item \textbf{Pré-processamento}: esta etapa é opcional. Aqui, o intuito é realizar quaisquer atividades prévias à extração de informações que interessam a etapa de extração de pontos de interesse. Tais atividades podem ser processos de melhoria na qualidade das medições (filtragem de ruídos), segmentação da área de interesse, obtenção da medição equivalente em 2D, deslocamentos nas posições das medições, etc.
	\item \textbf{Extração de pontos de interesse (\textit{feature extraction})}: o processo de \textit{feature extraction} consiste em, à partir das medições obtidas, selecionar e preparar os dados que o algoritmo de \textit{matching} utiliza pra fazer a autenticação do usuário.
	\item \textbf{Comparação (\textit{matching})}: consiste em utilizar os dados presentes no banco de dados do sistema de forma a compará-los com os dados fornecidos pela etapa de extração e, assim, calcular uma pontuação que indica o grau de similaridade entre eles.
	\item \textbf{Tomada de decisão}: de acordo com a pontuação obtida pelo \textit{matching}, o sistema utiliza um limiar para decidir se o usuário será aceito ou não. Caso a pontuação esteja dentro do limiar, o usuário será aceito; caso contrário, recusado.
\end{enumerate}

Previamente, é necessário que todas as pessoas que utilizariam o sistema façam o cadastramento de suas credenciais. Tal etapa é denominada \textbf{credenciamento}.

O processo acima descrito pode ser organizado das mais diversas maneiras, sendo algumas etapas fundidas, separadas ou recebendo nomenclaturas diferentes, variando de acordo com a literatura e objetivos da aplicação. As atividades realizadas, entretando, permanecem as mesmas, conforme \refFig{biometric_fluxogram}.

\figura{biometric_fluxogram}{Fluxograma básico de um sistema biométrico}{biometric_fluxogram}{width=1.0\textwidth}%

Deste ponto em diante, adotaremos o termo \acrfull{AFIS} para denotar sistemas de identificação de impressões digitais automatizados.

%%%%%%%%%%%%%%%%%%%%%%%%%%%%%%%%%%%%%%%%%%%%%%%%%%%%%%%%%%%%%%%%%%%%%%%%%%%%%%%%
%%%%%%%%%%%%%%%%%%%%%%%%%%%%%%%%%%%%%%%%%%%%%%%%%%%%%%%%%%%%%%%%%%%%%%%%%%%%%%%%
\subsection{Protocolos de autenticação}
Dependendo da política da aplicação, pode ser que apenas a apresentação de uma digital cadastrada em seus bancos de dados seja suficiente para a autenticação. Há aplicações, entretanto, que também exigem saber se o usuário é realmente quem alega ser. Partindo desta necessidade, há dois tipos de protocolos que definem quais informações formam a credencial de acesso de um usuário, sendo eles:

\begin{itemize}
	\item \textbf{Verificação (de identidade)}: no momento do engajamento, o usuário deve dizer ao \acrshort{AFIS} quem ele alega ser por meio de um PIN (ou seu CPF, ou seu RG, etc.) e, então, a sua impressão digital. O sistema, por sua vez, busca em seu banco de dados o PIN fornecido e checa se as medições da impressão digital do usuário são compatíveis com as medições relacionadas ao PIN no banco de dados. Dessa forma, o sistema faz uma busca (1:1). Desta forma, a verificação exige pelo menos duas credenciais e diminui o custo computacional ao buscar apenas a digital ligada ao PIN fornecido.
	\item \textbf{Identificação}: neste caso, no momento do engajamento, o usuário fornece apenas sua impressão digital. O \acrshort{AFIS}, por sua vez, irá buscar em seu banco de dados se alguma de suas medições é compatível com as medições do usuário e retornar uma lista contendo as identificações cujas pontuações ficaram dentro do limiar. Sendo assim, o processamento tende a ser mais custoso, pois a busca é (1:N).
\end{itemize}

A decisão a respeito de qual protocolo deve ser implementado depende do nível de segurança requerido, dos recursos computacionais disponíveis, tempo de resposta desejado, etc.

%%%%%%%%%%%%%%%%%%%%%%%%%%%%%%%%%%%%%%%%%%%%%%%%%%%%%%%%%%%%%%%%%%%%%%%%%%%%%%%%
%%%%%%%%%%%%%%%%%%%%%%%%%%%%%%%%%%%%%%%%%%%%%%%%%%%%%%%%%%%%%%%%%%%%%%%%%%%%%%%%
\subsection{Agente supervisor}
Outra ideia que, apesar de simples, merece atenção é a ausência ou presença de um agente (uma pessoa) que supervisiona as interações de um usuário com o sensor biométrico. Ambientes que possuem um agente supervisior são chamados de ``supervisionados''; ambientes nos quais não há um agente supervisor, ``não supervisionados''.

Deve-se destacar que ambientes não supervisionados tendem a ser mais suscetíveis a tentativas de ataque. Tal assunto, entretanto, será abordado na Seção \ref{spoofing}.

%%%%%%%%%%%%%%%%%%%%%%%%%%%%%%%%%%%%%%%%%%%%%%%%%%%%%%%%%%%%%%%%%%%%%%%%%%%%%%%%
%%%%%%%%%%%%%%%%%%%%%%%%%%%%%%%%%%%%%%%%%%%%%%%%%%%%%%%%%%%%%%%%%%%%%%%%%%%%%%%%
\subsection{Aquisição por deslize} \label{swipe}
Este método exige que o usuário deslize seu dedo sobre a superfície do sensor. Conforme o dedo é deslizado, um sequência de imagens é capturada. Essas imagens são, então, utilizadas para montar uma única imagem da digital completa.

Como a superfície de contato é pequena, o custo deste tipo de sensor tende a ser menor do que o de outros tipos, porém pode apresentar alta taxa de falhas de leitura \cite{presentation_attack_survey}.

Seu uso é muito comum em notebooks, como pode ser visto na \refFig{swipe_fingerprint}.

\figura{swipe_fingerprint}{Sensor biométrico de impressão digital com aquisição por deslize - OBTER IMAGEM PRÓPRIA!}{swipe_fingerprint}{width=.45\textwidth}%

%%%%%%%%%%%%%%%%%%%%%%%%%%%%%%%%%%%%%%%%%%%%%%%%%%%%%%%%%%%%%%%%%%%%%%%%%%%%%%%%
%%%%%%%%%%%%%%%%%%%%%%%%%%%%%%%%%%%%%%%%%%%%%%%%%%%%%%%%%%%%%%%%%%%%%%%%%%%%%%%%
\subsection{Aquisição baseada em toque} \label{touch}
O processo de aquisição de impressões digitais baseado em toque (mais conhecido como \textit{touch-based fingerprinting}) consiste em pressionar o dedo sobre a superfície do sensor e, dessa forma, a leitura das medições da impressão digital é realizada. É o tipo de aquisição mais comum nos sistemas atuais \cite{2D_3D_survey}.

Os dados extraídos neste processo representam a versão 2D da impressão digital, comumente representada em níveis de cinza. Um exemplo pode ser visto na \refFig{touch_based_fingerprint}.

\figura{touch_based_fingerprint}{Impressão digital obtida pelo método de aquisição baseado em toque - OBTER IMAGEM PRÓPRIA!}{touch_based_fingerprint}{width=.45\textwidth}%

Este processo, entretanto, implica em algumas desvantagens, que são inerentes ao modo com o qual o engajamento deve ser realizado.

A pele humana possui um certo grau de elasticidade. Ao realizar o contato entre o dedo e o sensor, a impressão digital sofre deformações. Dessa forma, as medições obtidas pelo sensor podem não ser suficientemente fieis à impressão digital original e sua replicabilidade é afetada. Outras fontes de deformações que também atuam são doenças de pele, umidade do ar, suor, etc.

É importante ressaltar que este método também pode representar uma grande falha de segurança, pois é possível que resíduos do dedo permaneçam na superfície de contato. Dessa forma, torna-se possível que um agente mal intencionado consiga obter uma digital válida à partir destes resíduos, principalmente em ambientes não supervisionados.

%%%%%%%%%%%%%%%%%%%%%%%%%%%%%%%%%%%%%%%%%%%%%%%%%%%%%%%%%%%%%%%%%%%%%%%%%%%%%%%%
%%%%%%%%%%%%%%%%%%%%%%%%%%%%%%%%%%%%%%%%%%%%%%%%%%%%%%%%%%%%%%%%%%%%%%%%%%%%%%%%
\subsection{Aquisição sem toque} \label{touchless}
O processo de aquisição de impressões digitais sem toque (mais conhecido como \textit{touchless fingerprinting}) consiste em posicionar o dedo em frente à um sensor que não exija contato. Tal sensor costuma fazer uso de uma (ou mais) câmera(s) de tal forma que as medições da impressão digital sejam obtidas à partir de imagens do dedo.

Devido a ausência de contato físico com a superfície do sensor, este método não deforma a impressão digital e é menos afetado por fatores como sujeira, umidade e demais condições da pele.

Duas abordagens são comumente utilizadas para a aquisição sem toque. A aquisição 2D é feita com o uso de apenas uma câmera. A imagem do dedo é capturada por apenas um ângulo e essa imagem é utilizada para extrair a impressão digital do dedo. O exemplo na \refFig{touchless_fingerprint} seria a aquisição realizada. É importante notar que este método pode acabar não capturando toda a digital, pois apenas um ângulo do dedo é utilizado.

Também há a aquisição 3D, que, comumente, é feita com o uso de três ou cinco câmeras posicionadas ao redor do dedo (\refFig{3D_touchless_scheme}). Cada câmera faz a captura da imagem do dedo por um ângulo diferente e, dessa forma, a impressão digital é capturada por completo. Essas imagens são processadas e as partes da digital capturadas em cada imagem são sobrepostas de forma que a impressão digital completa é reconstruída. \refFig{touchless_fingerprint} e \refFig{touchless_fingerprint_side}, neste caso, representam, cada uma, um dos ângulos capturados.

\figura{touchless_fingerprint}{Imagem capturada por um sensor sem toque}{touchless_fingerprint}{width=.45\textwidth}%

\figura{3D_touchless_scheme}{Esquema de câmeras em um sistema de aquisição sem toque 3D - OBTER IMAGEM PRÓPRIA!}{3D_touchless_scheme}{width=.45\textwidth}%

\figura{touchless_fingerprint_side}{Imagem capturada por uma das câmeras de um sensor sem toque 3D}{touchless_fingerprint_side}{width=.45\textwidth}%

As abordagens mais utilizadas no processo de captura das imagens do dedo são descritas nas Seções \ref{reflection} e \ref{transmission}.

%%%%%%%%%%%%%%%%%%%%%%%%%%%%%%%%%%%%%%%%%%%%%%%%%%%%%%%%%%%%%%%%%%%%%%%%%%%%%%%%
\subsubsection{Sistema legado}
Sabe-se que o método de aquisição mais utilizado nos \acrshort{AFIS}s é o baseado em toque. Ao criar um novo método de aquisição (sem toque), torna-se inconveniente ter que recadastrar todas as credenciais já existentes no sistema. Sendo assim, é importante garantir que haja compatibilidade entre ambos.

Ainda, note que, no escopo desta discussão, apenas a etapa de aquisição está sendo alterada, sendo que as outras etapas também devem continuar sendo compatíveis com o novo método de aquisição. Nos \acrshort{AFIS}s atuais, costuma-se utilizar, nas etapas de \textit{feature extraction} e \textit{matching}, algoritmos concebidos para imagens obtidas por sistemas baseados em toque \cite{advances_biometrics}.

Com o objetivo de unir essas ideias, \acrshort{AFIS}s de aquisição sem toque costumam transformar suas representações de impressões digitais em equivalentes 2D de sistemas baseados em toque. Dessa forma, consegue-se conciliar os dados obtidos pelos dois métodos no mesmo sistema.

Além de lidar com os problemas intrínsecos da aquisição baseada em toque, sistemas legados se beneficiam da aquisição sem toque ao passo que este método também reduz a ocorrência de \cite{advances_biometrics}:

\begin{itemize}
	\item Derrapagem e borramento devido a umidade.
	\item Contatos impróprios causados por pele seca.
	\item Acúmulo de sujeira na superfície de aquisição.
	\item Diminuição da qualidade da imagem adquirida, devido ao desgaste da superfície de contato.
	\item Erros de medições causados pela diferença de temperatura entre a superfície de contato do dedo e a superfície de contato do sensor.
\end{itemize}

%%%%%%%%%%%%%%%%%%%%%%%%%%%%%%%%%%%%%%%%%%%%%%%%%%%%%%%%%%%%%%%%%%%%%%%%%%%%%%%%
\subsubsection{Imageamento por Reflexão} \label{reflection}
Este método, também conhecido pelo termo em inglês \acrfull{RTFI}, baseia-se na maneira com que a luz é refletida na superfície de um dedo. Fontes luminosas são colocadas em frente a região da impressão digital em conjunto com sensores (comumente câmeras), que quantificam as reflexões.

Para que seja possível obter medições com constraste suficiente, é necessário que os seguintes requisitos sejam atendidos \cite{advances_biometrics}:

\begin{itemize}
		\item A pele do dedo deve refletir a maior parte da luz incidente sobre ela, ou seja, a escolha da fonte luminosa e até mesmo do tipo de luz utilizada devem ser tais que a pele humana absorva o mínimo possível.
		\item As quantidades de luz refletidas pelas dobras e vales devem ser diferentes.
		\item Fontes luminosas e sensores devem ser posicionados o mais perpendicular possível da superfície do dedo, para que penumbras sejam evitadas e para que os raios refletidos possam ser capturados pelo sensor.
\end{itemize}

%%%%%%%%%%%%%%%%%%%%%%%%%%%%%%%%%%%%%%%%%%%%%%%%%%%%%%%%%%%%%%%%%%%%%%%%%%%%%%%%
\subsubsection{Imageamento por Transmissão} \label{transmission}
Este método, também conhecido pelo termo em inglês \acrfull{TTFI}, baseia-se na maneira com que a luz atravessa o dedo.

As fontes luminosas são posicionadas acima da área na qual a unha deve estar e, abaixo do dedo, são posicionados os sensores (comumente câmeras) que quantificam a quantidade de luz que atravessa o dedo. Tal método assemelha-se a colocar uma lanterna acesa em um lado do dedo (fonte luminosa) e observar o lado oposto (os olhos são nossos sensores).